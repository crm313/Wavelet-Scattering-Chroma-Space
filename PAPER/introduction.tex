% introduction.tex
% 
% LaTeX document from NYU music technology master's thesis

\section{Introduction}
\label{introduction}

%%%%%%%%%%%%%%%%%%%%%%%%%%%%%%%%%%%%%%%

Chord sequences provide a succinct description of tonal music and are often written down on lead sheets for the use of
accompanists and improvisers.
Besides its original purpose in music education and transmission, the knowledge of
harmonic content has been leveraged in music information research to address higher-level
tasks, including cover song identification \cite{ellis2007identifying},
genre recognition \cite{perez2009genre}, and lyrics-to-audio alignment
\cite{mauch2012integrating}. We refer to the review of
\cite{mcvicar2014automatic} for a recent state of the art.

All evaluation metrics for automatic
chord estimation share the following basic property:
a chord label remains the same if all its components are jointly
transposed by one octave, be it upwards or downwards.
In order to comply with this requirement, the vast majority of existing
chord estimation systems rely on the chroma representation, \ie a 12-dimensional vector
derived from a log-frequency spectrum
(such as the constant-Q transform) by summing up all
frequency bands which share the same pitch class according to
the twelve-tone equal temperament.
However, it should be noted that the chroma representation is not
only invariant to octave transposition, but also to any permutation
of the chord factors --- an operation known in music theory
as inversion.
Although major and minor triads are unchanged by inversion,
some less common chords, such as augmented triads and minor seventh
tetrads, are conditional upon the position of the root.

\begin{figure}[b]
    \begin{center}
        \setlength{\unitlength}{1cm}
        \begin{picture}(8.2,1.5)
        \put(-0.1,-0.5){\includegraphics[width=8.2cm]{figs/sheet_music.png}}
        \end{picture}
    \end{center}
    \protect\caption{
Three possible voicings of the pitch class set
\pcset, resulting either in the chord \Amin
or \Csix. See text for details.
\label{fig:sheet-music}
}
\end{figure}

Figure \ref{fig:sheet-music} illustrates the importance of disambiguating inversions
when transcribing chords, which has previously been addressed by
\cite{mauch2010approximate}.
The first two voicings are identical up
to octave transposition of all the chord factors, and thus have the
same chord label \Amin.
In contrast, the third voicing is labeled as \Csix 
 in root position, although its third inversion would correspond
to the first voicing.

With the aim of improving automatic chord estimation (ACE) under fine-grained
evaluation metrics for large chord vocabularies (157 chord classes), this thesis introduces two feature extraction methods
that are invariant to octave transposition, yet sensitive to
chord inversion.
The first consists of computing a Haar wavelet transform of
the constant-Q spectrum along the octave variable and storing
the absolute values of the resulting coefficients at all scales
and positions.
The second iterates the Haar wavelet modulus nonlinear operator
over increasing scales, until reaching the full extent of the
constant-Q spectrum.
Both methods build upon the large chord vocabulary ACE software of
\cite{cho2013mirex}, which holds state-of-the-art performance on
the McGill Billboard dataset \cite{burgoyne2011}.

Section 2 of this thesis reviews the basic components of automatic chord estimation systems, describing in particular the multiband chroma features, as
introduced by \cite{cho2013mirex}, and their integration into a multi-stream
hidden Markov model.
Section 3 reviews the scattering transform as introduced by Mallat in \cite{mallat2012group}.
Section 4 defines the Haar wavelet transform across octaves
of the constant-Q spectrum and the deep Haar scattering transform, and compares these new representations with the multiband chroma.
Section 5 presents and discusses the experimental setup along with the
evaluation metrics for chord estimation accuracy.
Section 6 presents the results of large vocabulary chord estimation comparing all three feature extraction methods and Section 7 provides extended analysis at all points throughout the chord recognition system. Finally, Section 8 concludes this thesis.
