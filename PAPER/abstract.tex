% abstract.tex
% Chris Miller - crm313@nyu.edu
% 

\begin{abstract}

State of the art automatic chord recognition systems rely on multiband chroma representations,
Gaussian Mixture Model pattern matching, and Viterbi decoding.
This thesis explores the use of Haar wavelet transforms and scattering in place of multiband
chroma. Wavelets operating across octaves encode sums and differences in chroma bins at
different scales.
We describe both the Haar wavelet transform and deep wavelet scattering and develop an
efficient algorithm for their computation. Potential benefits of wavelet representations,
including stability to octave deformations, over multiband chroma are discussed.
Accuracy of wavelet representations used for chord recognition is analyzed over a large
vocabulary of chord qualities.

\end{abstract}

\newpage

\begin{center}
\textbf{Note} \\

\par \vskip \baselineskip
\noindent
This thesis is based in large part on a paper submitted to the 2016 ISMIR conference and currently pending review entitled \emph{Wavelet Scattering For Automatic Chord Estimation} by Chris Miller, Vincent Lostanlen, St\'ephane Mallat, and Juan Bello.

\end{center}